% Options for packages loaded elsewhere
% Options for packages loaded elsewhere
\PassOptionsToPackage{unicode}{hyperref}
\PassOptionsToPackage{hyphens}{url}
\PassOptionsToPackage{dvipsnames,svgnames,x11names}{xcolor}
%
\documentclass[
  letterpaper,
]{article}
\usepackage{xcolor}
\usepackage[margin=1in]{geometry}
\usepackage{amsmath,amssymb}
\setcounter{secnumdepth}{5}
\usepackage{iftex}
\ifPDFTeX
  \usepackage[T1]{fontenc}
  \usepackage[utf8]{inputenc}
  \usepackage{textcomp} % provide euro and other symbols
\else % if luatex or xetex
  \usepackage{unicode-math} % this also loads fontspec
  \defaultfontfeatures{Scale=MatchLowercase}
  \defaultfontfeatures[\rmfamily]{Ligatures=TeX,Scale=1}
\fi
\usepackage{lmodern}
\ifPDFTeX\else
  % xetex/luatex font selection
\fi
% Use upquote if available, for straight quotes in verbatim environments
\IfFileExists{upquote.sty}{\usepackage{upquote}}{}
\IfFileExists{microtype.sty}{% use microtype if available
  \usepackage[]{microtype}
  \UseMicrotypeSet[protrusion]{basicmath} % disable protrusion for tt fonts
}{}
\makeatletter
\@ifundefined{KOMAClassName}{% if non-KOMA class
  \IfFileExists{parskip.sty}{%
    \usepackage{parskip}
  }{% else
    \setlength{\parindent}{0pt}
    \setlength{\parskip}{6pt plus 2pt minus 1pt}}
}{% if KOMA class
  \KOMAoptions{parskip=half}}
\makeatother
% Make \paragraph and \subparagraph free-standing
\makeatletter
\ifx\paragraph\undefined\else
  \let\oldparagraph\paragraph
  \renewcommand{\paragraph}{
    \@ifstar
      \xxxParagraphStar
      \xxxParagraphNoStar
  }
  \newcommand{\xxxParagraphStar}[1]{\oldparagraph*{#1}\mbox{}}
  \newcommand{\xxxParagraphNoStar}[1]{\oldparagraph{#1}\mbox{}}
\fi
\ifx\subparagraph\undefined\else
  \let\oldsubparagraph\subparagraph
  \renewcommand{\subparagraph}{
    \@ifstar
      \xxxSubParagraphStar
      \xxxSubParagraphNoStar
  }
  \newcommand{\xxxSubParagraphStar}[1]{\oldsubparagraph*{#1}\mbox{}}
  \newcommand{\xxxSubParagraphNoStar}[1]{\oldsubparagraph{#1}\mbox{}}
\fi
\makeatother


\usepackage{longtable,booktabs,array}
\usepackage{calc} % for calculating minipage widths
% Correct order of tables after \paragraph or \subparagraph
\usepackage{etoolbox}
\makeatletter
\patchcmd\longtable{\par}{\if@noskipsec\mbox{}\fi\par}{}{}
\makeatother
% Allow footnotes in longtable head/foot
\IfFileExists{footnotehyper.sty}{\usepackage{footnotehyper}}{\usepackage{footnote}}
\makesavenoteenv{longtable}
\usepackage{graphicx}
\makeatletter
\newsavebox\pandoc@box
\newcommand*\pandocbounded[1]{% scales image to fit in text height/width
  \sbox\pandoc@box{#1}%
  \Gscale@div\@tempa{\textheight}{\dimexpr\ht\pandoc@box+\dp\pandoc@box\relax}%
  \Gscale@div\@tempb{\linewidth}{\wd\pandoc@box}%
  \ifdim\@tempb\p@<\@tempa\p@\let\@tempa\@tempb\fi% select the smaller of both
  \ifdim\@tempa\p@<\p@\scalebox{\@tempa}{\usebox\pandoc@box}%
  \else\usebox{\pandoc@box}%
  \fi%
}
% Set default figure placement to htbp
\def\fps@figure{htbp}
\makeatother





\setlength{\emergencystretch}{3em} % prevent overfull lines

\providecommand{\tightlist}{%
  \setlength{\itemsep}{0pt}\setlength{\parskip}{0pt}}



 


\usepackage{booktabs}
\usepackage{longtable}
\usepackage{float}
\usepackage{needspace}
\renewenvironment{abstract}{\par\bigskip\noindent\textbf{Abstract}\par\smallskip}{\par\bigskip}
\makeatletter
\@ifpackageloaded{caption}{}{\usepackage{caption}}
\AtBeginDocument{%
\ifdefined\contentsname
  \renewcommand*\contentsname{Table of contents}
\else
  \newcommand\contentsname{Table of contents}
\fi
\ifdefined\listfigurename
  \renewcommand*\listfigurename{List of Figures}
\else
  \newcommand\listfigurename{List of Figures}
\fi
\ifdefined\listtablename
  \renewcommand*\listtablename{List of Tables}
\else
  \newcommand\listtablename{List of Tables}
\fi
\ifdefined\figurename
  \renewcommand*\figurename{Figure}
\else
  \newcommand\figurename{Figure}
\fi
\ifdefined\tablename
  \renewcommand*\tablename{Table}
\else
  \newcommand\tablename{Table}
\fi
}
\@ifpackageloaded{float}{}{\usepackage{float}}
\floatstyle{ruled}
\@ifundefined{c@chapter}{\newfloat{codelisting}{h}{lop}}{\newfloat{codelisting}{h}{lop}[chapter]}
\floatname{codelisting}{Listing}
\newcommand*\listoflistings{\listof{codelisting}{List of Listings}}
\makeatother
\makeatletter
\makeatother
\makeatletter
\@ifpackageloaded{caption}{}{\usepackage{caption}}
\@ifpackageloaded{subcaption}{}{\usepackage{subcaption}}
\makeatother
\usepackage{bookmark}
\IfFileExists{xurl.sty}{\usepackage{xurl}}{} % add URL line breaks if available
\urlstyle{same}
\hypersetup{
  colorlinks=true,
  linkcolor={blue},
  filecolor={Maroon},
  citecolor={Blue},
  urlcolor={Blue},
  pdfcreator={LaTeX via pandoc}}


\author{}
\date{}
\begin{document}

\renewcommand*\contentsname{Table of contents}
{
\hypersetup{linkcolor=}
\setcounter{tocdepth}{3}
\tableofcontents
}

\section*{Appendix B: Methodological
Details}\label{appendix-b-methodological-details}
\addcontentsline{toc}{section}{Appendix B: Methodological Details}

\subsection{B.1 Structural Equation Modeling
Framework}\label{b.1-structural-equation-modeling-framework}

Structural Equation Modeling combines confirmatory factor analysis (CFA)
with path analysis. The key advantage is simultaneous estimation of:

\begin{enumerate}
\def\labelenumi{\arabic{enumi}.}
\tightlist
\item
  Measurement relationships (indicators → latents)
\item
  Structural relationships (latents → latents)
\end{enumerate}

\subsubsection{Notation}\label{notation}

\begin{itemize}
\tightlist
\item
  \(\eta\): Endogenous latent variables
\item
  \(\xi\): Exogenous latent variables
\item
  \(y\): Observed indicators for endogenous latents
\item
  \(x\): Observed indicators for exogenous latents
\item
  \(\Lambda_y\), \(\Lambda_x\): Factor loading matrices
\item
  \(B\): Structural coefficients among endogenous latents
\item
  \(\Gamma\): Structural coefficients from exogenous to endogenous
\item
  \(\Psi\): Covariance matrix of structural errors
\item
  \(\Theta_\varepsilon\), \(\Theta_\delta\): Measurement error
  covariances
\end{itemize}

\subsubsection{Model Equations}\label{model-equations}

\textbf{Measurement Model}: \[
y = \Lambda_y \eta + \varepsilon
\] \[
x = \Lambda_x \xi + \delta
\]

\textbf{Structural Model}: \[
\eta = B\eta + \Gamma\xi + \zeta
\]

\subsection{B.2 Model Identification}\label{b.2-model-identification}

For model identification, we require:

\begin{enumerate}
\def\labelenumi{\arabic{enumi}.}
\tightlist
\item
  At least 2 indicators per latent variable (preferably 3+)
\item
  Scale setting for each latent (fix first loading to 1 or fix variance
  to 1)
\item
  Positive degrees of freedom: \(df = \frac{p(p+1)}{2} - q > 0\)
\end{enumerate}

where \(p\) is the number of observed variables and \(q\) is the number
of free parameters.

\subsection{B.3 Estimation Method}\label{b.3-estimation-method}

Maximum likelihood estimation minimizes the fitting function:

\[
F_{ML} = \ln|\Sigma(\theta)| + \text{tr}(S\Sigma(\theta)^{-1}) - \ln|S| - p
\]

where \(\Sigma(\theta)\) is the model-implied covariance matrix, \(S\)
is the sample covariance matrix, and \(p\) is the number of observed
variables.

\subsection{B.4 Model Fit Assessment}\label{b.4-model-fit-assessment}

We evaluate model fit using multiple indices:

\begin{longtable}[]{@{}lll@{}}
\toprule\noalign{}
Index & Good Fit & Acceptable Fit \\
\midrule\noalign{}
\endhead
\bottomrule\noalign{}
\endlastfoot
CFI & ≥ 0.95 & ≥ 0.90 \\
TLI & ≥ 0.95 & ≥ 0.90 \\
RMSEA & ≤ 0.05 & ≤ 0.08 \\
SRMR & ≤ 0.05 & ≤ 0.08 \\
\end{longtable}

\subsection{B.5 Alternative
Specifications}\label{b.5-alternative-specifications}

We test multiple model specifications:

\begin{enumerate}
\def\labelenumi{\arabic{enumi}.}
\tightlist
\item
  \textbf{Full model}: All proposed indicators
\item
  \textbf{Reduced model}: Minimal indicators per factor
\item
  \textbf{With covariates}: Adding population, severity, experience
\item
  \textbf{Subset models}: State vs.~local governments
\end{enumerate}

\subsection{B.6 Software and
Replication}\label{b.6-software-and-replication}

Analysis is conducted using:

\begin{itemize}
\tightlist
\item
  Python 3.10+
\item
  semopy package for SEM estimation
\item
  pandas, numpy for data manipulation
\item
  matplotlib for visualization
\end{itemize}

Replication materials are available in the project repository.




\end{document}
