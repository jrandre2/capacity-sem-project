% Options for packages loaded elsewhere
% Options for packages loaded elsewhere
\PassOptionsToPackage{unicode}{hyperref}
\PassOptionsToPackage{hyphens}{url}
\PassOptionsToPackage{dvipsnames,svgnames,x11names}{xcolor}
%
\documentclass[
  letterpaper,
]{article}
\usepackage{xcolor}
\usepackage[margin=1in]{geometry}
\usepackage{amsmath,amssymb}
\setcounter{secnumdepth}{5}
\usepackage{iftex}
\ifPDFTeX
  \usepackage[T1]{fontenc}
  \usepackage[utf8]{inputenc}
  \usepackage{textcomp} % provide euro and other symbols
\else % if luatex or xetex
  \usepackage{unicode-math} % this also loads fontspec
  \defaultfontfeatures{Scale=MatchLowercase}
  \defaultfontfeatures[\rmfamily]{Ligatures=TeX,Scale=1}
\fi
\usepackage{lmodern}
\ifPDFTeX\else
  % xetex/luatex font selection
\fi
% Use upquote if available, for straight quotes in verbatim environments
\IfFileExists{upquote.sty}{\usepackage{upquote}}{}
\IfFileExists{microtype.sty}{% use microtype if available
  \usepackage[]{microtype}
  \UseMicrotypeSet[protrusion]{basicmath} % disable protrusion for tt fonts
}{}
\makeatletter
\@ifundefined{KOMAClassName}{% if non-KOMA class
  \IfFileExists{parskip.sty}{%
    \usepackage{parskip}
  }{% else
    \setlength{\parindent}{0pt}
    \setlength{\parskip}{6pt plus 2pt minus 1pt}}
}{% if KOMA class
  \KOMAoptions{parskip=half}}
\makeatother
% Make \paragraph and \subparagraph free-standing
\makeatletter
\ifx\paragraph\undefined\else
  \let\oldparagraph\paragraph
  \renewcommand{\paragraph}{
    \@ifstar
      \xxxParagraphStar
      \xxxParagraphNoStar
  }
  \newcommand{\xxxParagraphStar}[1]{\oldparagraph*{#1}\mbox{}}
  \newcommand{\xxxParagraphNoStar}[1]{\oldparagraph{#1}\mbox{}}
\fi
\ifx\subparagraph\undefined\else
  \let\oldsubparagraph\subparagraph
  \renewcommand{\subparagraph}{
    \@ifstar
      \xxxSubParagraphStar
      \xxxSubParagraphNoStar
  }
  \newcommand{\xxxSubParagraphStar}[1]{\oldsubparagraph*{#1}\mbox{}}
  \newcommand{\xxxSubParagraphNoStar}[1]{\oldsubparagraph{#1}\mbox{}}
\fi
\makeatother


\usepackage{longtable,booktabs,array}
\usepackage{calc} % for calculating minipage widths
% Correct order of tables after \paragraph or \subparagraph
\usepackage{etoolbox}
\makeatletter
\patchcmd\longtable{\par}{\if@noskipsec\mbox{}\fi\par}{}{}
\makeatother
% Allow footnotes in longtable head/foot
\IfFileExists{footnotehyper.sty}{\usepackage{footnotehyper}}{\usepackage{footnote}}
\makesavenoteenv{longtable}
\usepackage{graphicx}
\makeatletter
\newsavebox\pandoc@box
\newcommand*\pandocbounded[1]{% scales image to fit in text height/width
  \sbox\pandoc@box{#1}%
  \Gscale@div\@tempa{\textheight}{\dimexpr\ht\pandoc@box+\dp\pandoc@box\relax}%
  \Gscale@div\@tempb{\linewidth}{\wd\pandoc@box}%
  \ifdim\@tempb\p@<\@tempa\p@\let\@tempa\@tempb\fi% select the smaller of both
  \ifdim\@tempa\p@<\p@\scalebox{\@tempa}{\usebox\pandoc@box}%
  \else\usebox{\pandoc@box}%
  \fi%
}
% Set default figure placement to htbp
\def\fps@figure{htbp}
\makeatother





\setlength{\emergencystretch}{3em} % prevent overfull lines

\providecommand{\tightlist}{%
  \setlength{\itemsep}{0pt}\setlength{\parskip}{0pt}}



 


\usepackage{booktabs}
\usepackage{longtable}
\usepackage{float}
\usepackage{needspace}
\renewenvironment{abstract}{\par\bigskip\noindent\textbf{Abstract}\par\smallskip}{\par\bigskip}
\makeatletter
\@ifpackageloaded{caption}{}{\usepackage{caption}}
\AtBeginDocument{%
\ifdefined\contentsname
  \renewcommand*\contentsname{Table of contents}
\else
  \newcommand\contentsname{Table of contents}
\fi
\ifdefined\listfigurename
  \renewcommand*\listfigurename{List of Figures}
\else
  \newcommand\listfigurename{List of Figures}
\fi
\ifdefined\listtablename
  \renewcommand*\listtablename{List of Tables}
\else
  \newcommand\listtablename{List of Tables}
\fi
\ifdefined\figurename
  \renewcommand*\figurename{Figure}
\else
  \newcommand\figurename{Figure}
\fi
\ifdefined\tablename
  \renewcommand*\tablename{Table}
\else
  \newcommand\tablename{Table}
\fi
}
\@ifpackageloaded{float}{}{\usepackage{float}}
\floatstyle{ruled}
\@ifundefined{c@chapter}{\newfloat{codelisting}{h}{lop}}{\newfloat{codelisting}{h}{lop}[chapter]}
\floatname{codelisting}{Listing}
\newcommand*\listoflistings{\listof{codelisting}{List of Listings}}
\makeatother
\makeatletter
\makeatother
\makeatletter
\@ifpackageloaded{caption}{}{\usepackage{caption}}
\@ifpackageloaded{subcaption}{}{\usepackage{subcaption}}
\makeatother
\usepackage{bookmark}
\IfFileExists{xurl.sty}{\usepackage{xurl}}{} % add URL line breaks if available
\urlstyle{same}
\hypersetup{
  colorlinks=true,
  linkcolor={blue},
  filecolor={Maroon},
  citecolor={Blue},
  urlcolor={Blue},
  pdfcreator={LaTeX via pandoc}}


\author{}
\date{}
\begin{document}

\renewcommand*\contentsname{Table of contents}
{
\hypersetup{linkcolor=}
\setcounter{tocdepth}{3}
\tableofcontents
}

\section*{Appendix A: Data
Description}\label{appendix-a-data-description}
\addcontentsline{toc}{section}{Appendix A: Data Description}

\subsection{A.1 Data Sources}\label{a.1-data-sources}

\subsubsection{Quarterly Performance Reports
(QPR)}\label{quarterly-performance-reports-qpr}

The QPR data is extracted from HUD's DRGR (Disaster Recovery Grant
Reporting) system. This administrative data captures:

\begin{itemize}
\tightlist
\item
  Fund allocations by appropriation
\item
  Obligation, disbursement, and expenditure amounts
\item
  Activity types (housing, infrastructure, economic development, etc.)
\item
  Quarterly reporting periods
\end{itemize}

\subsubsection{Population Data}\label{population-data}

Population figures are obtained from:

\begin{itemize}
\tightlist
\item
  U.S. Census Bureau decennial census (2000, 2010, 2020)
\item
  American Community Survey estimates for intercensal years
\end{itemize}

\subsubsection{Disaster Severity}\label{disaster-severity}

The disaster severity index is constructed from:

\begin{itemize}
\tightlist
\item
  FEMA declared disaster data
\item
  Individual Assistance (IA) and Public Assistance (PA) obligations
\item
  Number of affected counties
\end{itemize}

\subsubsection{Experience Indicators}\label{experience-indicators}

Grantee experience measures are computed from:

\begin{itemize}
\tightlist
\item
  Years since first CDBG-DR grant
\item
  Number of prior disaster grants managed
\item
  Cumulative prior obligated dollars
\end{itemize}

\subsection{A.2 Variable Construction}\label{a.2-variable-construction}

\subsubsection{Financial Ratios}\label{financial-ratios}

\[
\text{Ratio\_disbursed\_to\_obligated} = \frac{\text{Total Disbursed}}{\text{Total Obligated}}
\]

\[
\text{Ratio\_expended\_to\_disbursed} = \frac{\text{Total Expended}}{\text{Total Disbursed}}
\]

\subsubsection{Duration Measures}\label{duration-measures}

Duration is computed as the number of months from first obligation to
reaching 95\% expenditure of the final obligated amount.
Log-transformation is applied to address right-skewness:

\[
\text{Duration\_log} = \ln(\text{Duration\_months})
\]

\subsubsection{Spending Consistency}\label{spending-consistency}

The coefficient of variation (CV) measures spending consistency:

\[
\text{Spending\_CV} = \frac{\sigma_{\text{quarterly}}}{\mu_{\text{quarterly}}}
\]

\subsection{A.3 Sample
Characteristics}\label{a.3-sample-characteristics}

{[}Include sample characteristics tables{]}

\subsection{A.4 Data Cleaning}\label{a.4-data-cleaning}

The following cleaning steps were applied:

\begin{enumerate}
\def\labelenumi{\arabic{enumi}.}
\tightlist
\item
  Removed total rows (aggregated across quarters)
\item
  Filtered to grantees with at least 4 quarters of reporting
\item
  Excluded records with zero obligated amounts
\item
  Winsorized extreme ratio values at 1st and 99th percentiles
\end{enumerate}




\end{document}
