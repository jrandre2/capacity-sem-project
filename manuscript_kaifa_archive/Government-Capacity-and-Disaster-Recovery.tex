% Options for packages loaded elsewhere
% Options for packages loaded elsewhere
\PassOptionsToPackage{unicode}{hyperref}
\PassOptionsToPackage{hyphens}{url}
\PassOptionsToPackage{dvipsnames,svgnames,x11names}{xcolor}
%
\documentclass[
  letterpaper,
  DIV=11,
  numbers=noendperiod]{scrreprt}
\usepackage{xcolor}
\usepackage[margin=1in]{geometry}
\usepackage{amsmath,amssymb}
\setcounter{secnumdepth}{5}
\usepackage{iftex}
\ifPDFTeX
  \usepackage[T1]{fontenc}
  \usepackage[utf8]{inputenc}
  \usepackage{textcomp} % provide euro and other symbols
\else % if luatex or xetex
  \usepackage{unicode-math} % this also loads fontspec
  \defaultfontfeatures{Scale=MatchLowercase}
  \defaultfontfeatures[\rmfamily]{Ligatures=TeX,Scale=1}
\fi
\usepackage{lmodern}
\ifPDFTeX\else
  % xetex/luatex font selection
\fi
% Use upquote if available, for straight quotes in verbatim environments
\IfFileExists{upquote.sty}{\usepackage{upquote}}{}
\IfFileExists{microtype.sty}{% use microtype if available
  \usepackage[]{microtype}
  \UseMicrotypeSet[protrusion]{basicmath} % disable protrusion for tt fonts
}{}
\makeatletter
\@ifundefined{KOMAClassName}{% if non-KOMA class
  \IfFileExists{parskip.sty}{%
    \usepackage{parskip}
  }{% else
    \setlength{\parindent}{0pt}
    \setlength{\parskip}{6pt plus 2pt minus 1pt}}
}{% if KOMA class
  \KOMAoptions{parskip=half}}
\makeatother
% Make \paragraph and \subparagraph free-standing
\makeatletter
\ifx\paragraph\undefined\else
  \let\oldparagraph\paragraph
  \renewcommand{\paragraph}{
    \@ifstar
      \xxxParagraphStar
      \xxxParagraphNoStar
  }
  \newcommand{\xxxParagraphStar}[1]{\oldparagraph*{#1}\mbox{}}
  \newcommand{\xxxParagraphNoStar}[1]{\oldparagraph{#1}\mbox{}}
\fi
\ifx\subparagraph\undefined\else
  \let\oldsubparagraph\subparagraph
  \renewcommand{\subparagraph}{
    \@ifstar
      \xxxSubParagraphStar
      \xxxSubParagraphNoStar
  }
  \newcommand{\xxxSubParagraphStar}[1]{\oldsubparagraph*{#1}\mbox{}}
  \newcommand{\xxxSubParagraphNoStar}[1]{\oldsubparagraph{#1}\mbox{}}
\fi
\makeatother


\usepackage{longtable,booktabs,array}
\usepackage{calc} % for calculating minipage widths
% Correct order of tables after \paragraph or \subparagraph
\usepackage{etoolbox}
\makeatletter
\patchcmd\longtable{\par}{\if@noskipsec\mbox{}\fi\par}{}{}
\makeatother
% Allow footnotes in longtable head/foot
\IfFileExists{footnotehyper.sty}{\usepackage{footnotehyper}}{\usepackage{footnote}}
\makesavenoteenv{longtable}
\usepackage{graphicx}
\makeatletter
\newsavebox\pandoc@box
\newcommand*\pandocbounded[1]{% scales image to fit in text height/width
  \sbox\pandoc@box{#1}%
  \Gscale@div\@tempa{\textheight}{\dimexpr\ht\pandoc@box+\dp\pandoc@box\relax}%
  \Gscale@div\@tempb{\linewidth}{\wd\pandoc@box}%
  \ifdim\@tempb\p@<\@tempa\p@\let\@tempa\@tempb\fi% select the smaller of both
  \ifdim\@tempa\p@<\p@\scalebox{\@tempa}{\usebox\pandoc@box}%
  \else\usebox{\pandoc@box}%
  \fi%
}
% Set default figure placement to htbp
\def\fps@figure{htbp}
\makeatother


% definitions for citeproc citations
\NewDocumentCommand\citeproctext{}{}
\NewDocumentCommand\citeproc{mm}{%
  \begingroup\def\citeproctext{#2}\cite{#1}\endgroup}
\makeatletter
 % allow citations to break across lines
 \let\@cite@ofmt\@firstofone
 % avoid brackets around text for \cite:
 \def\@biblabel#1{}
 \def\@cite#1#2{{#1\if@tempswa , #2\fi}}
\makeatother
\newlength{\cslhangindent}
\setlength{\cslhangindent}{1.5em}
\newlength{\csllabelwidth}
\setlength{\csllabelwidth}{3em}
\newenvironment{CSLReferences}[2] % #1 hanging-indent, #2 entry-spacing
 {\begin{list}{}{%
  \setlength{\itemindent}{0pt}
  \setlength{\leftmargin}{0pt}
  \setlength{\parsep}{0pt}
  % turn on hanging indent if param 1 is 1
  \ifodd #1
   \setlength{\leftmargin}{\cslhangindent}
   \setlength{\itemindent}{-1\cslhangindent}
  \fi
  % set entry spacing
  \setlength{\itemsep}{#2\baselineskip}}}
 {\end{list}}
\usepackage{calc}
\newcommand{\CSLBlock}[1]{\hfill\break\parbox[t]{\linewidth}{\strut\ignorespaces#1\strut}}
\newcommand{\CSLLeftMargin}[1]{\parbox[t]{\csllabelwidth}{\strut#1\strut}}
\newcommand{\CSLRightInline}[1]{\parbox[t]{\linewidth - \csllabelwidth}{\strut#1\strut}}
\newcommand{\CSLIndent}[1]{\hspace{\cslhangindent}#1}



\setlength{\emergencystretch}{3em} % prevent overfull lines

\providecommand{\tightlist}{%
  \setlength{\itemsep}{0pt}\setlength{\parskip}{0pt}}



 


\usepackage{booktabs}
\usepackage{longtable}
\usepackage{float}
\usepackage{needspace}
\renewenvironment{abstract}{\par\bigskip\noindent\textbf{Abstract}\par\smallskip}{\par\bigskip}
\KOMAoption{captions}{tableheading}
\makeatletter
\@ifpackageloaded{bookmark}{}{\usepackage{bookmark}}
\makeatother
\makeatletter
\@ifpackageloaded{caption}{}{\usepackage{caption}}
\AtBeginDocument{%
\ifdefined\contentsname
  \renewcommand*\contentsname{Table of contents}
\else
  \newcommand\contentsname{Table of contents}
\fi
\ifdefined\listfigurename
  \renewcommand*\listfigurename{List of Figures}
\else
  \newcommand\listfigurename{List of Figures}
\fi
\ifdefined\listtablename
  \renewcommand*\listtablename{List of Tables}
\else
  \newcommand\listtablename{List of Tables}
\fi
\ifdefined\figurename
  \renewcommand*\figurename{Figure}
\else
  \newcommand\figurename{Figure}
\fi
\ifdefined\tablename
  \renewcommand*\tablename{Table}
\else
  \newcommand\tablename{Table}
\fi
}
\@ifpackageloaded{float}{}{\usepackage{float}}
\floatstyle{ruled}
\@ifundefined{c@chapter}{\newfloat{codelisting}{h}{lop}}{\newfloat{codelisting}{h}{lop}[chapter]}
\floatname{codelisting}{Listing}
\newcommand*\listoflistings{\listof{codelisting}{List of Listings}}
\makeatother
\makeatletter
\makeatother
\makeatletter
\@ifpackageloaded{caption}{}{\usepackage{caption}}
\@ifpackageloaded{subcaption}{}{\usepackage{subcaption}}
\makeatother
\usepackage{bookmark}
\IfFileExists{xurl.sty}{\usepackage{xurl}}{} % add URL line breaks if available
\urlstyle{same}
\hypersetup{
  pdftitle={Government Capacity and Disaster Recovery},
  pdfauthor={{[}Author Name{]}},
  colorlinks=true,
  linkcolor={blue},
  filecolor={Maroon},
  citecolor={Blue},
  urlcolor={Blue},
  pdfcreator={LaTeX via pandoc}}


\title{Government Capacity and Disaster Recovery}
\usepackage{etoolbox}
\makeatletter
\providecommand{\subtitle}[1]{% add subtitle to \maketitle
  \apptocmd{\@title}{\par {\large #1 \par}}{}{}
}
\makeatother
\subtitle{A Structural Equation Model Analysis of CDBG-DR Program
Performance}
\author{{[}Author Name{]}}
\date{}
\begin{document}
\maketitle

\renewcommand*\contentsname{Table of contents}
{
\hypersetup{linkcolor=}
\setcounter{tocdepth}{2}
\tableofcontents
}

\bookmarksetup{startatroot}

\chapter{Government Capacity and Disaster
Recovery}\label{government-capacity-and-disaster-recovery}

\bookmarksetup{startatroot}

\chapter*{Abstract}\label{abstract}
\addcontentsline{toc}{chapter}{Abstract}

\markboth{Abstract}{Abstract}

This study examines the relationship between government capacity and
disaster recovery outcomes using Structural Equation Modeling (SEM). We
analyze Quarterly Performance Report (QPR) data from the Community
Development Block Grant-Disaster Recovery (CDBG-DR) program to assess
how administrative capacity affects the timeliness and effectiveness of
post-disaster fund expenditure. Our latent variable model captures
government capacity through financial management ratios while measuring
recovery outcomes through duration, completion rates, and spending
consistency. Results indicate {[}summarize key findings{]}. These
findings have implications for disaster preparedness policy and the
allocation of federal recovery resources.

\textbf{Keywords:} government capacity, disaster recovery, structural
equation modeling, CDBG-DR, public administration

\bookmarksetup{startatroot}

\chapter{Introduction}\label{introduction}

Disasters impose substantial costs on communities, and effective
recovery depends critically on the administrative capacity of
implementing governments. The Community Development Block Grant-Disaster
Recovery (CDBG-DR) program, administered by the U.S. Department of
Housing and Urban Development (HUD), is a major vehicle for federal
disaster recovery assistance. However, significant variation exists in
how quickly and effectively grantees expend allocated funds.

This study addresses a fundamental question: Does government
administrative capacity affect disaster recovery outcomes? We
operationalize capacity through observable financial management
indicators and assess its relationship to recovery timeliness and
effectiveness using Structural Equation Modeling.

\section{Research Questions}\label{research-questions}

\begin{enumerate}
\def\labelenumi{\arabic{enumi}.}
\tightlist
\item
  How does government capacity, measured through fund management
  metrics, relate to recovery outcomes?
\item
  Do the relationships between capacity and outcomes differ between
  state and local governments?
\item
  How do external factors (population, disaster severity, prior
  experience) moderate these relationships?
\end{enumerate}

\section{Contribution}\label{contribution}

This research contributes to the disaster recovery literature by:

\begin{itemize}
\tightlist
\item
  Developing a latent variable framework for measuring government
  capacity
\item
  Utilizing comprehensive QPR data across multiple disaster events and
  grantees
\item
  Applying SEM to simultaneously estimate measurement and structural
  relationships
\item
  Comparing capacity effects across government types
\end{itemize}

\bookmarksetup{startatroot}

\chapter{Literature Review}\label{literature-review}

\section{Government Capacity}\label{government-capacity}

The concept of government capacity has received extensive attention in
public administration scholarship. Capacity encompasses the resources,
skills, and organizational structures that enable effective policy
implementation (Wu et al., 2015). In the disaster context, capacity
includes preparedness, resource mobilization, and administrative
efficiency.

\section{Disaster Recovery
Performance}\label{disaster-recovery-performance}

Recovery performance has been conceptualized along multiple dimensions
including speed, equity, and completeness (Olshansky et al., 2006).
Prior research has documented substantial variation in recovery
timelines and outcomes across communities.

\section{CDBG-DR Program}\label{cdbg-dr-program}

The CDBG-DR program provides flexible block grants to states, cities,
and counties recovering from presidentially-declared disasters. Grantees
submit Quarterly Performance Reports documenting fund obligation and
expenditure, providing a rich data source for analyzing recovery
dynamics.

\bookmarksetup{startatroot}

\chapter{Data}\label{data}

\section{Data Sources}\label{data-sources}

Our analysis draws on:

\begin{enumerate}
\def\labelenumi{\arabic{enumi}.}
\tightlist
\item
  \textbf{Quarterly Performance Reports (QPR)}: Fund obligation,
  disbursement, and expenditure data by grantee, disaster, and activity
  type
\item
  \textbf{Population data}: Grantee jurisdiction populations from Census
  Bureau
\item
  \textbf{Disaster severity}: Composite index based on damage
  assessments
\item
  \textbf{Grantee experience}: Prior CDBG-DR grants administered
\end{enumerate}

\section{Sample Construction}\label{sample-construction}

The analysis sample includes {[}describe sample{]}. We exclude
{[}describe exclusions{]} to ensure adequate observation periods for
computing outcome measures.

\section{Variable Definitions}\label{variable-definitions}

\subsection{Capacity Indicators}\label{capacity-indicators}

\begin{itemize}
\tightlist
\item
  \textbf{Disbursement ratio}: Funds disbursed to recipients relative to
  obligated funds
\item
  \textbf{Expenditure ratio}: Funds expended by recipients relative to
  disbursed funds
\end{itemize}

\subsection{Outcome Indicators}\label{outcome-indicators}

\begin{itemize}
\tightlist
\item
  \textbf{Duration}: Months to reach 95\% expenditure of final obligated
  amount
\item
  \textbf{Spending CV}: Coefficient of variation of quarterly
  expenditures
\end{itemize}

\subsection{Covariates}\label{covariates}

\begin{itemize}
\tightlist
\item
  \textbf{Population}: Jurisdiction population (log-transformed and
  scaled)
\item
  \textbf{Severity}: Disaster severity composite index (scaled)
\item
  \textbf{Experience}: Prior grant management experience index (scaled)
\end{itemize}

\bookmarksetup{startatroot}

\chapter{Methodology}\label{methodology}

\section{Structural Equation
Modeling}\label{structural-equation-modeling}

We employ Structural Equation Modeling to simultaneously estimate:

\begin{enumerate}
\def\labelenumi{\arabic{enumi}.}
\tightlist
\item
  \textbf{Measurement models} linking observed indicators to latent
  constructs
\item
  \textbf{Structural paths} relating latent capacity to latent outcomes
\end{enumerate}

The general SEM can be written:

\[
\eta = B\eta + \Gamma\xi + \zeta
\]

where \(\eta\) represents endogenous latent variables, \(\xi\)
represents exogenous variables, \(B\) captures relationships among
endogenous variables, \(\Gamma\) captures effects of exogenous on
endogenous variables, and \(\zeta\) represents structural disturbances.

\section{Model Specification}\label{model-specification}

Our primary model specifies:

\textbf{Measurement Model (Capacity)}: \[
\text{gov\_cap} =\!\!\sim \text{Ratio\_disbursed} + \text{Ratio\_expended}
\]

\textbf{Measurement Model (Outcome)}: \[
\text{recovery\_outcome} =\!\!\sim \text{Duration\_log} + \text{Spending\_CV}
\]

\textbf{Structural Model}: \[
\text{recovery\_outcome} \sim \text{gov\_cap}
\]

\section{Estimation}\label{estimation}

Models are estimated using maximum likelihood via the \texttt{semopy}
Python package. We assess model fit using standard indices (CFI, TLI,
RMSEA, SRMR) against conventional thresholds.

\section{Robustness Checks}\label{robustness-checks}

We conduct robustness analyses including:

\begin{itemize}
\tightlist
\item
  Alternative model specifications
\item
  Subset analyses by government type
\item
  Sample sensitivity tests
\item
  Covariate inclusion/exclusion
\end{itemize}

\bookmarksetup{startatroot}

\chapter{Results}\label{results}

\section{Descriptive Statistics}\label{descriptive-statistics}

{[}Include descriptive statistics table{]}

\section{Model Fit}\label{model-fit}

{[}Include model fit table{]}

\section{Parameter Estimates}\label{parameter-estimates}

{[}Include parameter estimates table{]}

\section{Subset Analyses}\label{subset-analyses}

{[}Discuss state vs.~local differences{]}

\section{Robustness Checks}\label{robustness-checks-1}

{[}Discuss robustness results{]}

\bookmarksetup{startatroot}

\chapter{Discussion}\label{discussion}

\section{Key Findings}\label{key-findings}

{[}Summarize main findings{]}

\section{Theoretical Implications}\label{theoretical-implications}

{[}Discuss implications for capacity theory{]}

\section{Policy Implications}\label{policy-implications}

{[}Discuss implications for CDBG-DR administration and disaster
preparedness{]}

\section{Limitations}\label{limitations}

{[}Discuss limitations{]}

\section{Future Research}\label{future-research}

{[}Suggest future directions{]}

\bookmarksetup{startatroot}

\chapter{Conclusion}\label{conclusion}

{[}Concluding summary{]}

\bookmarksetup{startatroot}

\chapter*{References}\label{references}
\addcontentsline{toc}{chapter}{References}

\markboth{References}{References}

\cleardoublepage
\phantomsection
\addcontentsline{toc}{part}{Appendices}
\appendix

\chapter*{Appendix A: Data
Description}\label{appendix-a-data-description}
\addcontentsline{toc}{chapter}{Appendix A: Data Description}

\markboth{Appendix A: Data Description}{Appendix A: Data Description}

\section*{A.1 Data Sources}\label{a.1-data-sources}
\addcontentsline{toc}{section}{A.1 Data Sources}

\markright{A.1 Data Sources}

\subsection*{Quarterly Performance Reports
(QPR)}\label{quarterly-performance-reports-qpr}
\addcontentsline{toc}{subsection}{Quarterly Performance Reports (QPR)}

The QPR data is extracted from HUD's DRGR (Disaster Recovery Grant
Reporting) system. This administrative data captures:

\begin{itemize}
\tightlist
\item
  Fund allocations by appropriation
\item
  Obligation, disbursement, and expenditure amounts
\item
  Activity types (housing, infrastructure, economic development, etc.)
\item
  Quarterly reporting periods
\end{itemize}

\subsection*{Population Data}\label{population-data}
\addcontentsline{toc}{subsection}{Population Data}

Population figures are obtained from:

\begin{itemize}
\tightlist
\item
  U.S. Census Bureau decennial census (2000, 2010, 2020)
\item
  American Community Survey estimates for intercensal years
\end{itemize}

\subsection*{Disaster Severity}\label{disaster-severity}
\addcontentsline{toc}{subsection}{Disaster Severity}

The disaster severity index is constructed from:

\begin{itemize}
\tightlist
\item
  FEMA declared disaster data
\item
  Individual Assistance (IA) and Public Assistance (PA) obligations
\item
  Number of affected counties
\end{itemize}

\subsection*{Experience Indicators}\label{experience-indicators}
\addcontentsline{toc}{subsection}{Experience Indicators}

Grantee experience measures are computed from:

\begin{itemize}
\tightlist
\item
  Years since first CDBG-DR grant
\item
  Number of prior disaster grants managed
\item
  Cumulative prior obligated dollars
\end{itemize}

\section*{A.2 Variable Construction}\label{a.2-variable-construction}
\addcontentsline{toc}{section}{A.2 Variable Construction}

\markright{A.2 Variable Construction}

\subsection*{Financial Ratios}\label{financial-ratios}
\addcontentsline{toc}{subsection}{Financial Ratios}

\[
\text{Ratio\_disbursed\_to\_obligated} = \frac{\text{Total Disbursed}}{\text{Total Obligated}}
\]

\[
\text{Ratio\_expended\_to\_disbursed} = \frac{\text{Total Expended}}{\text{Total Disbursed}}
\]

\subsection*{Duration Measures}\label{duration-measures}
\addcontentsline{toc}{subsection}{Duration Measures}

Duration is computed as the number of months from first obligation to
reaching 95\% expenditure of the final obligated amount.
Log-transformation is applied to address right-skewness:

\[
\text{Duration\_log} = \ln(\text{Duration\_months})
\]

\subsection*{Spending Consistency}\label{spending-consistency}
\addcontentsline{toc}{subsection}{Spending Consistency}

The coefficient of variation (CV) measures spending consistency:

\[
\text{Spending\_CV} = \frac{\sigma_{\text{quarterly}}}{\mu_{\text{quarterly}}}
\]

\section*{A.3 Sample Characteristics}\label{a.3-sample-characteristics}
\addcontentsline{toc}{section}{A.3 Sample Characteristics}

\markright{A.3 Sample Characteristics}

{[}Include sample characteristics tables{]}

\section*{A.4 Data Cleaning}\label{a.4-data-cleaning}
\addcontentsline{toc}{section}{A.4 Data Cleaning}

\markright{A.4 Data Cleaning}

The following cleaning steps were applied:

\begin{enumerate}
\def\labelenumi{\arabic{enumi}.}
\tightlist
\item
  Removed total rows (aggregated across quarters)
\item
  Filtered to grantees with at least 4 quarters of reporting
\item
  Excluded records with zero obligated amounts
\item
  Winsorized extreme ratio values at 1st and 99th percentiles
\end{enumerate}

\chapter*{Appendix B: Methodological
Details}\label{appendix-b-methodological-details}
\addcontentsline{toc}{chapter}{Appendix B: Methodological Details}

\markboth{Appendix B: Methodological Details}{Appendix B: Methodological
Details}

\section*{B.1 Structural Equation Modeling
Framework}\label{b.1-structural-equation-modeling-framework}
\addcontentsline{toc}{section}{B.1 Structural Equation Modeling
Framework}

\markright{B.1 Structural Equation Modeling Framework}

Structural Equation Modeling combines confirmatory factor analysis (CFA)
with path analysis. The key advantage is simultaneous estimation of:

\begin{enumerate}
\def\labelenumi{\arabic{enumi}.}
\tightlist
\item
  Measurement relationships (indicators → latents)
\item
  Structural relationships (latents → latents)
\end{enumerate}

\subsection*{Notation}\label{notation}
\addcontentsline{toc}{subsection}{Notation}

\begin{itemize}
\tightlist
\item
  \(\eta\): Endogenous latent variables
\item
  \(\xi\): Exogenous latent variables
\item
  \(y\): Observed indicators for endogenous latents
\item
  \(x\): Observed indicators for exogenous latents
\item
  \(\Lambda_y\), \(\Lambda_x\): Factor loading matrices
\item
  \(B\): Structural coefficients among endogenous latents
\item
  \(\Gamma\): Structural coefficients from exogenous to endogenous
\item
  \(\Psi\): Covariance matrix of structural errors
\item
  \(\Theta_\varepsilon\), \(\Theta_\delta\): Measurement error
  covariances
\end{itemize}

\subsection*{Model Equations}\label{model-equations}
\addcontentsline{toc}{subsection}{Model Equations}

\textbf{Measurement Model}: \[
y = \Lambda_y \eta + \varepsilon
\] \[
x = \Lambda_x \xi + \delta
\]

\textbf{Structural Model}: \[
\eta = B\eta + \Gamma\xi + \zeta
\]

\section*{B.2 Model Identification}\label{b.2-model-identification}
\addcontentsline{toc}{section}{B.2 Model Identification}

\markright{B.2 Model Identification}

For model identification, we require:

\begin{enumerate}
\def\labelenumi{\arabic{enumi}.}
\tightlist
\item
  At least 2 indicators per latent variable (preferably 3+)
\item
  Scale setting for each latent (fix first loading to 1 or fix variance
  to 1)
\item
  Positive degrees of freedom: \(df = \frac{p(p+1)}{2} - q > 0\)
\end{enumerate}

where \(p\) is the number of observed variables and \(q\) is the number
of free parameters.

\section*{B.3 Estimation Method}\label{b.3-estimation-method}
\addcontentsline{toc}{section}{B.3 Estimation Method}

\markright{B.3 Estimation Method}

Maximum likelihood estimation minimizes the fitting function:

\[
F_{ML} = \ln|\Sigma(\theta)| + \text{tr}(S\Sigma(\theta)^{-1}) - \ln|S| - p
\]

where \(\Sigma(\theta)\) is the model-implied covariance matrix, \(S\)
is the sample covariance matrix, and \(p\) is the number of observed
variables.

\section*{B.4 Model Fit Assessment}\label{b.4-model-fit-assessment}
\addcontentsline{toc}{section}{B.4 Model Fit Assessment}

\markright{B.4 Model Fit Assessment}

We evaluate model fit using multiple indices:

\begin{longtable}[]{@{}lll@{}}
\toprule\noalign{}
Index & Good Fit & Acceptable Fit \\
\midrule\noalign{}
\endhead
\bottomrule\noalign{}
\endlastfoot
CFI & ≥ 0.95 & ≥ 0.90 \\
TLI & ≥ 0.95 & ≥ 0.90 \\
RMSEA & ≤ 0.05 & ≤ 0.08 \\
SRMR & ≤ 0.05 & ≤ 0.08 \\
\end{longtable}

\section*{B.5 Alternative
Specifications}\label{b.5-alternative-specifications}
\addcontentsline{toc}{section}{B.5 Alternative Specifications}

\markright{B.5 Alternative Specifications}

We test multiple model specifications:

\begin{enumerate}
\def\labelenumi{\arabic{enumi}.}
\tightlist
\item
  \textbf{Full model}: All proposed indicators
\item
  \textbf{Reduced model}: Minimal indicators per factor
\item
  \textbf{With covariates}: Adding population, severity, experience
\item
  \textbf{Subset models}: State vs.~local governments
\end{enumerate}

\section*{B.6 Software and
Replication}\label{b.6-software-and-replication}
\addcontentsline{toc}{section}{B.6 Software and Replication}

\markright{B.6 Software and Replication}

Analysis is conducted using:

\begin{itemize}
\tightlist
\item
  Python 3.10+
\item
  semopy package for SEM estimation
\item
  pandas, numpy for data manipulation
\item
  matplotlib for visualization
\end{itemize}

Replication materials are available in the project repository.

\chapter*{Appendix C: Robustness
Checks}\label{appendix-c-robustness-checks}
\addcontentsline{toc}{chapter}{Appendix C: Robustness Checks}

\markboth{Appendix C: Robustness Checks}{Appendix C: Robustness Checks}

\section*{C.1 Alternative Model
Specifications}\label{c.1-alternative-model-specifications}
\addcontentsline{toc}{section}{C.1 Alternative Model Specifications}

\markright{C.1 Alternative Model Specifications}

We compare multiple model specifications to assess sensitivity:

\begin{longtable}[]{@{}llll@{}}
\toprule\noalign{}
Model & Gov Capacity Indicators & Recovery Outcome Indicators & df \\
\midrule\noalign{}
\endhead
\bottomrule\noalign{}
\endlastfoot
Full & 3 (incl.~Timeliness) & 3 (incl.~Duration) & 8 \\
Reduced & 2 & 2 & 1 \\
Optimal V1 & 2 & 2 (log duration, CV) & 1 \\
Improved 3x3 & 3 (incl.~Startup Lag) & 3 (incl.~Time to 50\%) & 8 \\
\end{longtable}

{[}Include model comparison table{]}

\section*{C.2 Government Type
Subsets}\label{c.2-government-type-subsets}
\addcontentsline{toc}{section}{C.2 Government Type Subsets}

\markright{C.2 Government Type Subsets}

We estimate the model separately for:

\begin{enumerate}
\def\labelenumi{\arabic{enumi}.}
\tightlist
\item
  \textbf{All grantees} (N = {[}full sample{]})
\item
  \textbf{State governments} (N = {[}state sample{]})
\item
  \textbf{Local governments} (N = {[}local sample{]})
\end{enumerate}

{[}Include subset comparison table{]}

\section*{C.3 Sample Sensitivity}\label{c.3-sample-sensitivity}
\addcontentsline{toc}{section}{C.3 Sample Sensitivity}

\markright{C.3 Sample Sensitivity}

We vary the minimum quarters requirement to assess sample sensitivity:

\begin{longtable}[]{@{}lllll@{}}
\toprule\noalign{}
Min Quarters & N & CFI & RMSEA & Capacity Effect \\
\midrule\noalign{}
\endhead
\bottomrule\noalign{}
\endlastfoot
3 & & & & \\
4 & & & & \\
5 & & & & \\
6 & & & & \\
\end{longtable}

\section*{C.4 Covariate Robustness}\label{c.4-covariate-robustness}
\addcontentsline{toc}{section}{C.4 Covariate Robustness}

\markright{C.4 Covariate Robustness}

We test sensitivity to covariate inclusion:

\begin{enumerate}
\def\labelenumi{\arabic{enumi}.}
\tightlist
\item
  \textbf{Baseline}: No covariates
\item
  \textbf{Population only}: Adding population control
\item
  \textbf{Full covariates}: Population, severity, and experience
\end{enumerate}

{[}Include covariate robustness table{]}

\section*{C.5 Standard Error
Robustness}\label{c.5-standard-error-robustness}
\addcontentsline{toc}{section}{C.5 Standard Error Robustness}

\markright{C.5 Standard Error Robustness}

We compare standard error estimates:

\begin{enumerate}
\def\labelenumi{\arabic{enumi}.}
\tightlist
\item
  \textbf{ML standard errors}: Default maximum likelihood
\item
  \textbf{Bootstrap standard errors}: 500 replications
\item
  \textbf{Cluster-robust adjustments}: Clustering by disaster event
\end{enumerate}

{[}Include SE comparison table{]}

\section*{C.6 Sensitivity
Conclusions}\label{c.6-sensitivity-conclusions}
\addcontentsline{toc}{section}{C.6 Sensitivity Conclusions}

\markright{C.6 Sensitivity Conclusions}

{[}Summarize robustness findings{]}

\phantomsection\label{refs}
\begin{CSLReferences}{1}{0}
\bibitem[\citeproctext]{ref-recovery_ref_1}
Olshansky, R. B., Johnson, L. A., \& Topping, K. C. (2006). Rebuilding
communities following disaster: Lessons from kobe and los angeles.
\emph{Built Environment}, \emph{32}(4), 354--374.

\bibitem[\citeproctext]{ref-capacity_ref_1}
Wu, X., Ramesh, M., \& Howlett, M. (2015). Policy capacity: A conceptual
framework for understanding policy competences and capabilities.
\emph{Policy and Society}, \emph{34}(3-4), 165--171.
\url{https://doi.org/10.1016/j.polsoc.2015.09.001}

\end{CSLReferences}




\end{document}
